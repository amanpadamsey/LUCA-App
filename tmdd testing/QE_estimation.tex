% Options for packages loaded elsewhere
\PassOptionsToPackage{unicode}{hyperref}
\PassOptionsToPackage{hyphens}{url}
\PassOptionsToPackage{dvipsnames,svgnames,x11names}{xcolor}
%
\documentclass[
  letterpaper,
  DIV=11,
  numbers=noendperiod]{scrartcl}

\usepackage{amsmath,amssymb}
\usepackage{lmodern}
\usepackage{iftex}
\ifPDFTeX
  \usepackage[T1]{fontenc}
  \usepackage[utf8]{inputenc}
  \usepackage{textcomp} % provide euro and other symbols
\else % if luatex or xetex
  \usepackage{unicode-math}
  \defaultfontfeatures{Scale=MatchLowercase}
  \defaultfontfeatures[\rmfamily]{Ligatures=TeX,Scale=1}
\fi
% Use upquote if available, for straight quotes in verbatim environments
\IfFileExists{upquote.sty}{\usepackage{upquote}}{}
\IfFileExists{microtype.sty}{% use microtype if available
  \usepackage[]{microtype}
  \UseMicrotypeSet[protrusion]{basicmath} % disable protrusion for tt fonts
}{}
\makeatletter
\@ifundefined{KOMAClassName}{% if non-KOMA class
  \IfFileExists{parskip.sty}{%
    \usepackage{parskip}
  }{% else
    \setlength{\parindent}{0pt}
    \setlength{\parskip}{6pt plus 2pt minus 1pt}}
}{% if KOMA class
  \KOMAoptions{parskip=half}}
\makeatother
\usepackage{xcolor}
\setlength{\emergencystretch}{3em} % prevent overfull lines
\setcounter{secnumdepth}{-\maxdimen} % remove section numbering
% Make \paragraph and \subparagraph free-standing
\ifx\paragraph\undefined\else
  \let\oldparagraph\paragraph
  \renewcommand{\paragraph}[1]{\oldparagraph{#1}\mbox{}}
\fi
\ifx\subparagraph\undefined\else
  \let\oldsubparagraph\subparagraph
  \renewcommand{\subparagraph}[1]{\oldsubparagraph{#1}\mbox{}}
\fi


\providecommand{\tightlist}{%
  \setlength{\itemsep}{0pt}\setlength{\parskip}{0pt}}\usepackage{longtable,booktabs,array}
\usepackage{calc} % for calculating minipage widths
% Correct order of tables after \paragraph or \subparagraph
\usepackage{etoolbox}
\makeatletter
\patchcmd\longtable{\par}{\if@noskipsec\mbox{}\fi\par}{}{}
\makeatother
% Allow footnotes in longtable head/foot
\IfFileExists{footnotehyper.sty}{\usepackage{footnotehyper}}{\usepackage{footnote}}
\makesavenoteenv{longtable}
\usepackage{graphicx}
\makeatletter
\def\maxwidth{\ifdim\Gin@nat@width>\linewidth\linewidth\else\Gin@nat@width\fi}
\def\maxheight{\ifdim\Gin@nat@height>\textheight\textheight\else\Gin@nat@height\fi}
\makeatother
% Scale images if necessary, so that they will not overflow the page
% margins by default, and it is still possible to overwrite the defaults
% using explicit options in \includegraphics[width, height, ...]{}
\setkeys{Gin}{width=\maxwidth,height=\maxheight,keepaspectratio}
% Set default figure placement to htbp
\makeatletter
\def\fps@figure{htbp}
\makeatother

\KOMAoption{captions}{tableheading}
\makeatletter
\makeatother
\makeatletter
\makeatother
\makeatletter
\@ifpackageloaded{caption}{}{\usepackage{caption}}
\AtBeginDocument{%
\ifdefined\contentsname
  \renewcommand*\contentsname{Table of contents}
\else
  \newcommand\contentsname{Table of contents}
\fi
\ifdefined\listfigurename
  \renewcommand*\listfigurename{List of Figures}
\else
  \newcommand\listfigurename{List of Figures}
\fi
\ifdefined\listtablename
  \renewcommand*\listtablename{List of Tables}
\else
  \newcommand\listtablename{List of Tables}
\fi
\ifdefined\figurename
  \renewcommand*\figurename{Figure}
\else
  \newcommand\figurename{Figure}
\fi
\ifdefined\tablename
  \renewcommand*\tablename{Table}
\else
  \newcommand\tablename{Table}
\fi
}
\@ifpackageloaded{float}{}{\usepackage{float}}
\floatstyle{ruled}
\@ifundefined{c@chapter}{\newfloat{codelisting}{h}{lop}}{\newfloat{codelisting}{h}{lop}[chapter]}
\floatname{codelisting}{Listing}
\newcommand*\listoflistings{\listof{codelisting}{List of Listings}}
\makeatother
\makeatletter
\@ifpackageloaded{caption}{}{\usepackage{caption}}
\@ifpackageloaded{subcaption}{}{\usepackage{subcaption}}
\makeatother
\makeatletter
\@ifpackageloaded{tcolorbox}{}{\usepackage[many]{tcolorbox}}
\makeatother
\makeatletter
\@ifundefined{shadecolor}{\definecolor{shadecolor}{rgb}{.97, .97, .97}}
\makeatother
\makeatletter
\makeatother
\ifLuaTeX
  \usepackage{selnolig}  % disable illegal ligatures
\fi
\IfFileExists{bookmark.sty}{\usepackage{bookmark}}{\usepackage{hyperref}}
\IfFileExists{xurl.sty}{\usepackage{xurl}}{} % add URL line breaks if available
\urlstyle{same} % disable monospaced font for URLs
\hypersetup{
  pdftitle={Testing Quenching efficiency for Quenching Antibody},
  pdfauthor={Aman Padamsey},
  colorlinks=true,
  linkcolor={blue},
  filecolor={Maroon},
  citecolor={Blue},
  urlcolor={Blue},
  pdfcreator={LaTeX via pandoc}}

\title{Testing Quenching efficiency for Quenching Antibody}
\author{Aman Padamsey}
\date{}

\begin{document}
\maketitle
\ifdefined\Shaded\renewenvironment{Shaded}{\begin{tcolorbox}[interior hidden, borderline west={3pt}{0pt}{shadecolor}, enhanced, boxrule=0pt, breakable, sharp corners, frame hidden]}{\end{tcolorbox}}\fi

\hypertarget{aim}{%
\subsection{Aim:}\label{aim}}

Determining the quenching efficiency of quenching antibody using
anti-IgG beads

\hypertarget{requirements}{%
\subsection{Requirements}\label{requirements}}

\begin{itemize}
\item
  \href{https://www.bangslabs.com/product-selection/816}{Quantum™ Simply
  Cellular® anti-Human IgG} (\emph{Catalog number 816)}
\item
  AF-488 labelled human antibody
\item
  \href{https://www.thermofisher.com/antibody/product/Alexa-Fluor-488-Antibody-Polyclonal/A-11094}{Alexa
  Fluor 488 Polyclonal Antibody} \emph{(ThermoFischer Catalog number
  A-11094)}
\item
  Cell medium intended for cellular experiment
\end{itemize}

\hypertarget{procedure}{%
\subsection{Procedure}\label{procedure}}

\begin{itemize}
\item
  Dilute labelled antibody to 200nM in cell media. (Volume needed
  \textasciitilde500µL)
\item
  Take bottles containing anti-IgG beads, shake the bottle to ensure a
  uniform suspension (Do not vortex or sonicate the stock bottle)
\item
  Split diluted antibody solution in 5 eppes (100µL in each eppe)
\item
  Add one drop from each of the bottles of beads into the eppes
  separately (labeled bottles 1 -- 4 and `B')
\item
  Incubate in the dark for 10 minutes at RT
\item
  Add 1ml of cell media buffer to each eppe after incubation and
  centrifuge at 2500 x G for 5 minutes
\item
  Discard (decant) supernatant
\item
  Repeat 1x more
\item
  Re-suspend beads in 1ml of cell media. Split each eppe in two (0.5 ml
  of re-suspended bead from each eppe in a new eppe)
\item
  Make up volume in each eppe to 1ml with cell media (add 0.5 ml cell
  media in each eppe)
\item
  Centrifuge as before
\item
  Re-suspend one half of the `stained-bead' eppes with 100 µL of the
  same concentration of quenching antibody as used/intended for cellular
  experiment \emph{(ideally \textgreater= 10µg/ml). Re-suspend}
  remaining half of the `stained-bead' eppes with 100µL of cell media
  only.
\item
  Incubate in fridge for same duration of time used for used/intended
  for cellular experiment \emph{(ideally \textgreater= 10min)}
\item
  Add 100µL of medium in each of the eppes to bring the volume to 200µL
  and read in FACS (plate reader for convenience.
\end{itemize}

\hypertarget{calculation}{%
\subsection{Calculation}\label{calculation}}

\begin{itemize}
\item
  Calculate MFI of each of the bead population (Geo-mean/Median)
\item
  Subtract the MFI of all of the bead populations with that of the
  background `B'
\item
  Use the following equation to calculate the QE for each of the bead
  population:

  \begin{quote}
  \(QE = (MFIunquenched - MFIquenched)/MFIunquenched\)
  \end{quote}
\item
  Ideally the QE for each bead population (\#1-4) should be the same and
  should be high (\textgreater0.85)
\end{itemize}



\end{document}
